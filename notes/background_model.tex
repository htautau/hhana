\documentclass{article}

\usepackage{mathenv}
\usepackage{amsmath}

\title{The Had-Had Background Model}

\begin{document}

\maketitle


\section{Definitions}

\begin{eqnarray*}
OS =& Opposite Sign: (tau1_charge * tau2_charge) == -1 \\
!OS =& Not Opposite Sign: (tau1_charge * tau2_charge) != -1 \\
SS =& Same Sign: (tau1_charge * tau2_charge) == +1 \\
OSFF =& Opposite sign, but both taus fail tau-ID requirement \\
SSFF =& Same sign, but both taus fail tau-ID requirement
\end{eqnarray*}

Four QCD models are defined as:

\begin{eqnarray*}
!OS QCD =& A_nos x [!OS Data - B_nos x !OS Z (embedded) - !OS Other MC] \\
SS QCD =& A_ss x [SS Data - B_ss x SS Z (embedded) - SS Other MC] \\
SSFF QCD =& A_ssff x [SSFF Data - B_ssff x SSFF Z (embedded) - SSFF Other MC] \\
OSFF QCD =& \frac{SS QCD}{SSFF QCD} x A_osff x [OSFF Data - B_osff x OSFF Z (embedded) - OSFF Other MC]
\end{eqnarray*}

\section{Description}

\begin{itemize}

\item Two major backgrounds are Ztautau and QCD. All others are minor.

\item Using embedded Ztautau, data-driven QCD and all else is MC.

\item Using the SS QCD model, $A_{ss}$ and $B_{ss}$ are determined via a fit of the
2D recounted tau number of tracks distribution (the so-called "TrackFit"):

\[
[B_ss x OS Z + SS QCD + OS Other MC] fitted to [OS Data]
\]

The normalization of "Other MC" is fixed

The two floating parameters are $A_{ss}$ and $B_{ss}$.
$A_{ss}$ is present in the SS QCD model.
$B_{ss}$ is present in *both* the SS QCD model scaling the SS Z subtraction, and OS Z.

The tracks have been recounted in a larger dR<0.6 cone to get a better handle on QCD.

\item The SS QCD model is used in the TrackFit to get the normalizations of QCD
and Z, but the !OS QCD model is used from here onwards.

$A_{nos}$ and $B_{nos}$ are determined from $A_{ss}$ and $B_{ss}$ such that:

\[
|SS QCD| == |!OS QCD| and |SS Z| == |!OS Z|
\]

i.e. the number of expected QCD and Z events between models are the same.

\item  Very simply, we make a fit to estimate how much QCD and Z we have in the OS data.
To perform the fit, we need a template for QCD and this comes from SS QCD.
Once we know the amount of QCD, we need a model for all of the kinematical variables
and this is !OS due to the larger statistics and improved modelling of the
low MMC mass.

\item The OSFF QCD model is used only as a shape systematic on the SS QCD model in
the TrackFit.

$A_{osff}$ and $A_{ssff}$ are set to 1.0 (reasonable approximation)
$B_{osff}$, $B_{ssff}$ are determined such that the number of expected QCD
events between models are the same.

\item The SS QCD model will be used as a shape systematic on the !OS QCD model in
the signal extraction.

\item Currently performing the TrackFit separately from the signal extraction.
The uncertainty on this fit is then a separate NP in the final fit in the
signal extraction.

\item The final model will involve performing the TrackFit (using SS QCD)
simultaneously with a fit of the BDT score distribution (using !OS QCD).

\item Currently, All NPs (excluding theory NPs) are treated as shape+norm
systematics (HistFactory::HistoSys).

\end{itemize}

\section{Known Problems}

\begin{itemize}
\item In the QCD template we need to fix the MC subtraction
  since HistFactory cannot handle the Z and MC subtraction. There is no
  mechanism for "subsamples" in HistFactory.

  We cannot handle a *floating subtraction* of Z in the QCD model using HistFactory.

  MC subtraction is anyway small, so ignoring the floating MC subtraction is
  potentially a safe approximation! However, it would be nice to have the
  possibility of treating this properly in the future.
\end{itemize}


\section{FAQ}

\begin{itemize}
    \item Why do you need to use SS QCD in the TrackFit and not !OS?

   The !OS QCD model includes taus with an even number of tracks and this
   biases the recounted track distribution. Since we are fitting to OS data we
   need to use SS QCD.

   \item Why use ALPGEN Z instead of embedding for trackfit

       Concerned about track distribution in embedding

       Efficiencies were originally derived using ALPGEN
\end{itemize}
